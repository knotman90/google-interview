\chapter{Modulo Problem}
\section{Informal problem statement}
In this problem we are given an function $F(i,j,k)$ and two integers $N$ and $P$, and we have to first find out what is the maximum value that function can result in, and then count the number of ways we can obtain such result.

Given two integers $1 \leq N \leq P $ determine the number of unordered triplets $(i,j,k)$, $1 \leq i,j,k \leq P$, s.t. the Eq. \ref{eq:modulo_problem:formula} is maximized.
\begin{equation}
	\label{eq:modulo_problem:formula}
	F(i,j,k)^N = ( ( (N \bmod{i}) \bmod{j} ) \bmod{k}) \bmod{N}
\end{equation}

\section{Example}
For example suppose $N = 4$ and $P = 5$, then a the following would all be valid triplets, because Eq. \ref{eq:modulo_problem:formula} evaluates to $\max\limits_{i,j,k}F(i,j,k)^N=1$ ( the maximum value this function ever reach) when we plug them in:
\begin{itemize}
	\item $(3,3,3) \Rightarrow ( ( (4 \bmod{3}) \bmod{3} ) \bmod{3}) \bmod{4} = 1 \bmod{3} \bmod{3} \bmod{4} = 1$
	\item $(3,3,4) \Rightarrow ( ( (4 \bmod{3}) \bmod{3} ) \bmod{4}) \bmod{4} = 1 \bmod{3} \bmod{4} \bmod{4} = 1$
	\item $(5,5,3) \Rightarrow ( ( (4 \bmod{5}) \bmod{5} ) \bmod{3}) \bmod{4} = 4 \bmod{5} \bmod{3} \bmod{4} = 4 \bmod{3} =1$
\end{itemize}

\section{Solution}
First, let's start by looking a bit closer to Eq. \ref{eq:modulo_problem:formula} and let define $M = \max\limits_{i,j,k}F(i,j,k)^N$.

We notice that (Let's focus on the case where $N > 1$) :

\begin{enumerate}
	\item $M \leq N$ because the innermost expression i.e. $(N \bmod i) \leq N$ irrespective of the value of $i$
	\item $M < N$ because $(x \bmod N)= < N$
	\item $M >=1$, by for instance plugging-in $i=j=k=N-1$
	\item Since $M \ge 1$ then  $( ( (N \bmod{i}) \bmod{j} ) \bmod{k}) \neq N$. Moreover $(N \bmod i) \leq N$ so $( ( (N \bmod{i}) \bmod{j} ) \bmod{k}) < N$. For these reasons we can drop the last modulo operation. It is going to make no difference for our final result.
\end{enumerate}

In order to find the exact value of $M$ let's try to figure out what the maximum value of the following Equation is: $N \bmod{i}$
Simply by sampling some values for a certain $N$ (let's say $10$ and $17$) we can get a sense of what $M$ is:

For $N=10$ the maximum value $M=4$ is obtained for $i=6$:
\begin{align*}
	(N \bmod{9}) &= 1  & (N \bmod{8}) &= 1 & (N \bmod{7}) &= 1 \\
	(N \bmod{6}) &= 1  & (N \bmod{5}) &= 1 & (N \bmod{4}) &= 1 \\
	(N \bmod{3}) &= 1  & (N \bmod{2}) &= 1 & (N \bmod{1}) &= 1 
\end{align*}
while for $N=17$, $M=8$ is obtained for $i=9$:
\begin{align*}
	(N \bmod{16}) &= 1  & (N \bmod{15}) &= 2 & (N \bmod{14}) &= 3 & (N \bmod{13}) &= 4 \\
	(N \bmod{12}) &= 5  & (N \bmod{11}) &= 6 & (N \bmod{10}) &= 7 & (N \bmod{9})  &= 8 \\
	(N \bmod{8})  &= 1  & (N \bmod{7})  &= 3 & (N \bmod{6})  &= 5 & (N \bmod{5})  &= 2 \\
	(N \bmod{4})  &= 1  & (N \bmod{3})  &= 2 & (N \bmod{2})  &= 1 & (N \bmod{1})  &= 0 
\end{align*}

From the previous examples it seems that the maximum is $ \floor{\frac{N}{2}}$ when $N$ is odd and $ \floor{\frac{N-1}{2}}$  when $N$ is even.


We can actually prove that this is always the case:
If $N$ is odd then:
\begin{itemize}
	\item $\floor{\frac{N}{2}} = \frac{N-1}{2}$
	\item if $i \leq \frac{N-1}{2}$ then $(N \bmod{i}) < \frac{N-1}{2}$
	\item if $i > \frac{N-1}{2}$ then $(N \bmod{i}) = N-i$. 
	\item $N-i$ is maximum when $i$ is smallest i.e. at $\frac{N-1}{2}+1$. In this case, $(N \bmod{i}) = N - (\frac{N-1}{2}+1) = \frac{N-1}{2} = \floor{\frac{N}{2}}$
\end{itemize}

If $N$ is even then:
\begin{itemize}
	\item if $i \leq \frac{N}{2}$ then $(N \bmod{i}) < \frac{N}{2}$
	\item if $i > \frac{N}{2}$ then $(N \bmod{i}) = N-i$. 
	\item $N-i$ is maximum when $i$ is smallest i.e. at $\frac{N}{2}+1$. In this case, $(N \bmod{i}) = N - (\frac{N}{2}+1) = \frac{N}{2}-1 = \floor{\frac{N-1}{2}}$
\end{itemize}

Now we that we know that the value of $M = \max\limits_{i,j,k}F(i,j,k)^N$ we can start conting the number of solutions to our problem.


The strategy in counting consist in trying to obtain $M$ during one of the modulo operations of the Eq. \ref{eq:modulo_problem:formula} and then propagate is thoughout the rest of the formula. If the result of $(N mod i) $ in Eq. \ref{eq:modulo_problem:formula} is $M$ then values of $j$ and $k$ must be larger than $M$. If $(N \bmod i) \neq M$ then we will have to let $N$ propagate to the next modulo operation and then the same resoning applied for the first modulo operation applies to the rest of the equation i.e. $( ( N \bmod{j} ) \bmod{k}) \bmod{N}$

Let $G_N= P-N$ be the count of numbers strictly greater than $N$ and  $G_M= P-M$ the count of numbers greater than $M$·
We can distinguish three distinc cases:
\begin{enumerate}
	\item $(N \bmod i)=M$. Then $j,k$ are forces to be greater than $M$. The number of possible solution arising from this case is $G_M^2$.
	\item $(N \bmod i)=N$ and  $(N \bmod j)=M$. There are $G_N$ possible values for $i$ one for $j$ and $G_M$ for $k$. The total count for this case is $G_N \times G_M$
	\item $(N \bmod i)=$N and  $(N \bmod j)=N$. There are $G_N$ possible values for $i$ and $j$ but only one for $k$. The total count for this case is $G_N^2$
\end{enumerate}

The solution to the problem is the sum of solutions for each indivial case: 
\begin{equation}
	\label{eq:modulo_problem:solution}
	G_M^2 + G_N^2 + G_N G_M 
\end{equation}


